\documentclass[11pt,a4paper,oneside,openright]{report}

\usepackage[bindingoffset=5mm,left=20mm,right=20mm,top=20mm,bottom=20mm,footskip=10mm]{geometry}
\usepackage[utf8x]{inputenc}
\usepackage{hyperref}
\usepackage[english]{babel}
\usepackage{listings}
\usepackage{subfiles}
\usepackage{longtable}
\usepackage{multirow}
\usepackage{ragged2e} 
\usepackage{cmap} % avoid fi ligatures in pdf file
\usepackage{amsthm} % example numbering
\usepackage{color}
%\usepackage{bold-extra} % for bold tt font. Remember to include bold-extra.sty file
\usepackage{graphicx}
\usepackage[yyyymmdd]{datetime}
\usepackage{float}

% style for code listing
\renewcommand{\familydefault}{\sfdefault}
\newtheorem{example}{Example}[chapter]  % example numbering
\lstset{language=C}                     % formatting for code listing
\lstset{basicstyle=\ttfamily,breaklines=true}
\definecolor{darkGreen}{rgb}{0,0.4,0}
\definecolor{mybrown}{rgb}{0.40,0.10,0.05}
\lstset{commentstyle=\color{darkGreen}}  % comments color
\lstset{keywordstyle=\color{blue}}       % keyword color
\lstset{stringstyle=\color{mybrown}}     % string color
\lstset{showstringspaces=false}          % don't mark spaces in strings

\renewcommand{\dateseparator}{-}

% command for turning indent back on after \flushleft
\newcommand{\indenton}{\RaggedRight\parindent=15pt}

% command for vertical space
\newcommand{\vspacesmall}{\vspace{3mm}}
\newcommand{\vspacebig}{\vspace{6mm}}

% style for code inlined in text:
\newcommand{\codei}[1]{\bfseries \ttfamily{#1}\normalfont}


\begin{document}

\begin{titlepage}
    \centering
   
    \null %empty box needed for vfill to work
    \vfill

   {\bfseries\Huge
    Quaternion.h
    \vspacesmall
    
    Quaternion extension for 
    \vspacesmall
        
    C++ vector class library 
    \vspacebig
        
   }        
    \vspacebig
    
   {\Large    
    Agner Fog
    \vspacebig
    
    \copyright\ \today. Apache license 2.0
   }
    
    \vfill
    
    \includegraphics[width=306pt]{freesoftwarelogo.jpg}
    \vfill
    
\end{titlepage}

\RaggedRight

\chapter{Introduction}\label{chap:Introduction}
Quaternions or hypercomplex numbers is a topic in theoretical algebra and quantum physics. Applications relating to 3-D geometry and electromagnetism are better served with the vector3d package to VCL.
\vspacesmall

The file quaternion.h provides classes, operators, and functions for 
calculations with quaternions. This is an extension to the Vector Class Library.
\vspacesmall

The classes listed below are defined. Common operators and functions are defined for these classes:

\begin {table}[H]
\caption{Quaternion classes}
\label{table:QuaternionClasses}
\begin{tabular}{|p{24mm}|p{20mm}|p{20mm}|p{22mm}|p{20mm}|p{28mm}|}
\hline
\bfseries Quaternion class & \bfseries Precision &  \bfseries Quaternion elements per vector & \bfseries Correspon-ding real vector class & \bfseries Total bits & \bfseries Recommended minimum \newline instruction set \\ \hline
Quaternion1f  & \centering single & \centering  1 & \centering Vec4f & \centering 128 & SSE2 \\ \hline
Quaternion1d  & \centering double & \centering 1 & \centering Vec4d & \centering 256 & AVX \\ \hline
\end{tabular}
\end{table}
\vspacebig



\section{Compiling} \label{Compiling}
The quaternion class extension to the Vector Class Library is compiled in the same way as the Vector Class Library itself. All x86 and x86-64 platforms are supported, including Windows, Linux, and Mac OS. 
The following C++ compilers can be used: Gnu, Clang, Microsoft, and Intel. 
See the Vector class library manual for further details.
\vspacesmall

This example shows how to use the quaternion classes:

\begin{example}
\label{example1}
\end{example} % frame disappears if I put this after end lstlisting
\begin{lstlisting}[frame=single]
// Example for quaternions
#include <stdio.h>
#include "vectorclass.h"  // vector class library
#include "quaternion.h"   // quaternion extension

// function to print quaternion
template <typename Q>
void printqx (const char * text, Q a) {
    auto aa = a.to_vector(); // get elements as real vector
    printf("\n%s ", text);   // print text
    printf("(%.3G,%.3G,%.3G,%.3G)", aa[0], aa[1], aa[2], aa[3]);
}

int main() {
    // define quaternions
    Quaternion1d a(1,2,3,4);   //  1 + 2*i + 3*j + 4*k
    Quaternion1d b(2,-3,-1,0); //  2 - 3*i - 1*j + 0*k
    Quaternion1d c = a + b;    // add quaternions
    Quaternion1d d = a * b;    // multiply quaternions

    // print results
    printqx("a = ", a);        // a = (1,2,3,4)
    printqx("b = ", b);        // b = (2,-3,-1,0)
    printqx("c = ", c);        // c = (3,-1,2,4)
    printqx("d = ", d);        // d = (11,5,-7,15)
}

\end{lstlisting}
\vspacesmall


\chapter{Constructing quaternions and loading data} 
\label{ConstructingQuaternions}

There are several ways to create quaternions and put data into them. These methods are listed here.
\vspacebig

\begin{tabular}{|p{25mm}|p{100mm}|}
\hline
\bfseries Method & default constructor \\ \hline
\bfseries Defined for & all quaternion classes \\ \hline
\bfseries Description & the quaternion is created but not initialized.\newline
The value is unpredictable \\ \hline
\bfseries Efficiency & good \\ \hline
\end{tabular}
\vspacesmall

\begin{lstlisting}[frame=none]
// Example:
quaternion1f a;    // creates a quaternion of four floats
\end{lstlisting}
\vspacebig


\begin{tabular}{|p{25mm}|p{100mm}|}
\hline
\bfseries Method & Construct from single real \\ \hline
\bfseries Defined for & all quaternion classes \\ \hline
\bfseries Description & The parameter defines the real part. The imaginary parts are zero. \\ \hline
\bfseries Efficiency & good \\ \hline
\end{tabular}
\vspacesmall

\begin{lstlisting}[frame=none]
// Example:
quaternion1d a(3);  // a = (3,0,0,0)
\end{lstlisting}
\vspacebig


\begin{tabular}{|p{25mm}|p{100mm}|}
\hline
\bfseries Method & Construct from one real and three imaginary parts \\ \hline
\bfseries Defined for & all quaternion classes \\ \hline
\bfseries Description & The parameters define the real and imaginary parts  \\ \hline
\bfseries Efficiency & good \\ \hline
\end{tabular}
\vspacesmall

\begin{lstlisting}[frame=none]
// Example:
quaternion1d a(1,2,3,4);  // a = (1,2,3,4)
\end{lstlisting}
\vspacebig


\begin{tabular}{|p{25mm}|p{100mm}|}
\hline
\bfseries Method & Construct from two complex numbers \\ \hline
\bfseries Defined for & all quaternion classes \\ \hline
\bfseries Description & The second parameter is post-multiplied by j \\ \hline
\bfseries Efficiency & good \\ \hline
\bfseries Implementation & complexvec1.h must be included before quaternion.h \\ \hline
\end{tabular}
\vspacesmall

\begin{lstlisting}[frame=none]
// Example:
Complex1d a(1,2);    // a = 1 + i*2
Complex1d b(3,4);    // b = 3 + i*4
Quaternion1d c(a,b); // c = a + b*j = 1 + i*2 + j*3 + k*4
\end{lstlisting}
\vspacebig

\begin{tabular}{|p{25mm}|p{100mm}|}
\hline
\bfseries Method & member function load(p) \\ \hline
\bfseries Defined for & all quaternion classes \\ \hline
\bfseries Description & Load data from array of same precision. 
Each real part must be followed by the corresponding three imaginary parts. \\ \hline
\bfseries Efficiency & good \\ \hline
\end{tabular}
\vspacesmall

\begin{lstlisting}[frame=none]
// Example:
double a[4] = {1,2,3,4};
Quaternion1d b;
b.load(a);  // b = (1,2,3,4)
\end{lstlisting}
\vspacebig


\begin{tabular}{|p{25mm}|p{100mm}|}
\hline
\bfseries Method & member function store(p) \\ \hline
\bfseries Defined for & all quaternion classes \\ \hline
\bfseries Description & Save data into array of same precision. 
Each real part is followed by the corresponding three imaginary parts. \\ \hline
\bfseries Efficiency & good \\ \hline
\end{tabular}
\vspacesmall

\begin{lstlisting}[frame=none]
// Example:
float a[4];
Quaternion1f b(1,2,3,4);
b.store(a);  // a = {1,2,3,4}
\end{lstlisting}
\vspacebig


\begin{tabular}{|p{25mm}|p{100mm}|}
\hline
\bfseries Method & member function real() \\ \hline
\bfseries Defined for & all quaternion classes \\ \hline
\bfseries Description & Get real part of quaternion \\ \hline
\bfseries Efficiency & good \\ \hline
\end{tabular}
\vspacesmall

\begin{lstlisting}[frame=none]
// Example:
Quaternion1d a(1,2,3,4);
double r = a.real();  // a = 1
\end{lstlisting}
\vspacebig


\begin{tabular}{|p{25mm}|p{100mm}|}
\hline
\bfseries Method & member function imag() \\ \hline
\bfseries Defined for & all quaternion classes \\ \hline
\bfseries Description & Get imaginary parts of quaternion. The real part is set to zero \\ \hline
\bfseries Efficiency & good \\ \hline
\end{tabular}
\vspacesmall

\begin{lstlisting}[frame=none]
// Example:
Quaternion1d a(1,2,3,4);
Quaternion1d im = a.imag();  // a = (0,2,3,4)
\end{lstlisting}
\vspacebig


\begin{tabular}{|p{25mm}|p{100mm}|}
\hline
\bfseries Method & member function get\_low() \\ \hline
\bfseries Defined for & all quaternion classes \\ \hline
\bfseries Description & Get the real and the first imaginary part (i) as a complex vector \\ \hline
\bfseries Efficiency & good \\ \hline
\bfseries Implementation & complexvec1.h must be included before quaternion.h \\ \hline
\end{tabular}
\vspacesmall

\begin{lstlisting}[frame=none]
// Example:
Quaternion1d a(1,2,3,4);
Complex1d b = a.get_low();  // b = (1,2)
\end{lstlisting}
\vspacebig


\begin{tabular}{|p{25mm}|p{100mm}|}
\hline
\bfseries Method & member function get\_high() \\ \hline
\bfseries Defined for & all quaternion classes \\ \hline
\bfseries Description & Get the last two imaginary parts (j and k) as a complex vector \\ \hline
\bfseries Efficiency & good \\ \hline
\bfseries Implementation & complexvec1.h must be included before quaternion.h \\ \hline
\end{tabular}
\vspacesmall

\begin{lstlisting}[frame=none]
// Example:
Quaternion1d a(1,2,3,4);
Complex1d b = a.get_low();  // b = (1,2)
Complex1d c = a.get_high(); // c = (3,4)
Quaternion1d d(b,c);        // d = (1,2,3,4)
\end{lstlisting}
\vspacebig



\chapter{Operators}\label{chap:Operators}

\begin{tabular}{|p{25mm}|p{100mm}|}
\hline
\bfseries Operator & + \\ \hline
\bfseries Defined for & all quaternion classes  \\ \hline
\bfseries Description & Add two quaternions, or one quaternion and one real scalar of the same precision \\ \hline
\bfseries Efficiency & good \\ \hline
\end{tabular}
\vspacesmall

\begin{lstlisting}[frame=none]
// Example:
Quaternion1d a(1,2,3,4);
Quaternion1d b(5,6,7,8);
Quaternion1d c = a + b;    // c = (6,8,10,12)
Quaternion1d d = a + 10.0; // d = (11,2,3,4)
\end{lstlisting}
\vspacebig


\begin{tabular}{|p{25mm}|p{100mm}|}
\hline
\bfseries Operator & - \\ \hline
\bfseries Defined for & all quaternion classes  \\ \hline
\bfseries Description & Subtract two quaternions, or one quaternion and one real scalar of the same precision \\ \hline
\bfseries Efficiency & good \\ \hline
\end{tabular}
\vspacesmall

\begin{lstlisting}[frame=none]
// Example:
Quaternion1d a(12,11,10,9);
Quaternion1d b(5,6,7,8);
Quaternion1d c = a - b;    // c = (7,5,3,1)
Quaternion1d d = a - 10.0; // d = (2,11,10,9)
\end{lstlisting}
\vspacebig


\begin{tabular}{|p{25mm}|p{100mm}|}
\hline
\bfseries Operator & * \\ \hline
\bfseries Defined for & all quaternion classes  \\ \hline
\bfseries Description & Multiply two quaternions, or one quaternion and one real scalar of the same precision. \newline 
Multiplication of quaternions is not commutative, i.e. a*b and b*a are not the same.
\\ \hline
\bfseries Efficiency & medium \\ \hline
\bfseries Accuracy & Quaternion multiplication involves the calculation of sums of products. Loss of precision may occur if the result is close to zero. \\ \hline
\end{tabular}
\vspacesmall

\begin{lstlisting}[frame=none]
// Example:
Quaternion1d a(1,2,3,4);
Quaternion1d b(5,6,7,8);
Quaternion1d c = a * b;    // c = (-60,12,30,24)
Quaternion1d d = b * a;    // d = (-60,20,14,32)
Quaternion1d e = a * 10.;  // e = (10,20,30,40)

\end{lstlisting}
\vspacebig


\begin{tabular}{|p{25mm}|p{100mm}|}
\hline
\bfseries Operator & / \\ \hline
\bfseries Defined for & all quaternion classes  \\ \hline
\bfseries Description & Divide two quaternions, or one quaternion and one real scalar of the same precision. \newline
Division is defined as a / b = a * reciprocal(b) \\ \hline
\bfseries Efficiency & medium \\ \hline
\bfseries Accuracy & Quaternion division involves the calculation of sums of products. Loss of precision may occur if the result is close to zero. \\ \hline
\end{tabular}
\vspacesmall

\begin{lstlisting}[frame=none]
// Example:
Quaternion1f a(7,9,-1,7);
Quaternion1f b(1,2,3,2);
Quaternion1f c = a / b;    // c = (2,1,-1,-2)
Quaternion1f d = c * b;    // d = (7,9,-1,7)
Quaternion1f e = b / 2.0f; // e = (0.5,1,1.5,1)
Quaternion1f f = 18.f / b; // f = (1,-2,-3,-2)
Quaternion1f g = f * b;    // g = (18,0,0,0)
\end{lstlisting}
\vspacebig


\begin{tabular}{|p{25mm}|p{100mm}|}
\hline
\bfseries Operator & $\sim$ \\ \hline
\bfseries Defined for & all quaternion classes  \\ \hline
\bfseries Description & Complex conjugate. The signs of the imaginary parts are inverted \\ \hline
\bfseries Efficiency & good \\ \hline
\end{tabular}
\vspacesmall

\begin{lstlisting}[frame=none]
// Example:
Quaternion1f a(1,2,3,4);
Quaternion1f b = ~ a;    // b = (1,-2,-3,-4)
\end{lstlisting}
\vspacebig


\begin{tabular}{|p{25mm}|p{100mm}|}
\hline
\bfseries Operator & == \\ \hline
\bfseries Defined for & all quaternion classes  \\ \hline
\bfseries Description & Compare for equality.\newline
The result is a boolean scalar. \\ \hline
\bfseries Efficiency & good \\ \hline
\end{tabular}
\vspacesmall

\begin{lstlisting}[frame=none]
// Example:
Quaternion1f a(1, 2,3,4);
Quaternion1f b(1,-2,3,4);
bool         c = (a == b);  // c = false
\end{lstlisting}
\vspacebig


\begin{tabular}{|p{25mm}|p{100mm}|}
\hline
\bfseries Operator & != \\ \hline
\bfseries Defined for & all quaternion classes  \\ \hline
\bfseries Description & Compare for not equal.\newline
The result is a boolean scalar. \\ \hline
\bfseries Efficiency & good \\ \hline
\end{tabular}
\vspacesmall

\begin{lstlisting}[frame=none]
// Example:
Quaternion1f a(1, 2,3,4);
Quaternion1f b(1,-2,3,4);
bool         c = (a != b);  // c = true
\end{lstlisting}
\vspacebig


\chapter{Mathematical functions}\label{chap:MathematicalFunctions}


\begin{tabular}{|p{25mm}|p{100mm}|}
\hline
\bfseries Function & abs \\ \hline
\bfseries Defined for & all quaternion classes  \\ \hline
\bfseries Description & Gives the norm as a scalar \\ \hline
\bfseries Efficiency & medium \\ \hline
\end{tabular}
\vspacesmall

\begin{lstlisting}[frame=none]
// Example:
Quaternion1f a(2,1,0,2);
double       b = abs(a); // b = 3
\end{lstlisting}
\vspacebig



\chapter{Other functions}\label{chap:OtherFunctions}


\begin{tabular}{|p{25mm}|p{100mm}|}
\hline
\bfseries Function & to\_vector \\ \hline
\bfseries Defined for & all quaternion classes \\ \hline
\bfseries Description & Convert to a vector of the real part and the three imaginary parts. \\ \hline
\bfseries Efficiency & good \\ \hline
\end{tabular}
\vspacesmall

\begin{lstlisting}[frame=none]
// Example:
Quaternion1d a(1,2,3,4);
Vec4d        b = a.to_vector(); // b = (1,2,3,4)
\end{lstlisting}
\vspacebig


\begin{tabular}{|p{25mm}|p{100mm}|}
\hline
\bfseries Function & select \\ \hline
\bfseries Defined for & all quaternion classes  \\ \hline
\bfseries Description & Choose between two quaternions. \\ \hline
\bfseries Efficiency & good \\ \hline
\end{tabular}
\vspacesmall

\begin{lstlisting}[frame=none]
// Example:
Quaternion1d a(1,2,3,4);
Quaternion1d b(5,6,7,8);
Quaternion1d c = select(true,a,b);  // c = (1,2,3,4)
Quaternion1d d = select(false,a,b); // d = (5,6,7,8)
\end{lstlisting}
\vspacebig


\end{document}
