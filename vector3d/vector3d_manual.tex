\documentclass[11pt,a4paper,oneside,openright]{report}

\usepackage[bindingoffset=5mm,left=20mm,right=20mm,top=20mm,bottom=20mm,footskip=10mm]{geometry}
\usepackage[utf8x]{inputenc}
\usepackage{hyperref}
\usepackage[english]{babel}
\usepackage{listings}
\usepackage{subfiles}
\usepackage{longtable}
\usepackage{multirow}
\usepackage{ragged2e} 
\usepackage{cmap} % avoid fi ligatures in pdf file
\usepackage{amsthm} % example numbering
\usepackage{color}
%\usepackage{bold-extra} % for bold tt font. Remember to include bold-extra.sty file
\usepackage{graphicx}
\usepackage[yyyymmdd]{datetime}
\usepackage{float}

% style for code listing
\renewcommand{\familydefault}{\sfdefault}
\renewcommand{\ttdefault}{pcr} % selects Courier font
\newtheorem{example}{Example}[chapter]  % example numbering
\lstset{language=C}                     % formatting for code listing
\lstset{basicstyle=\ttfamily,breaklines=true}
\definecolor{darkGreen}{rgb}{0,0.4,0}
\definecolor{mybrown}{rgb}{0.40,0.10,0.05}
\lstset{commentstyle=\color{darkGreen}}  % comments color
\lstset{keywordstyle=\color{blue}}       % keyword color
\lstset{stringstyle=\color{mybrown}}     % string color
\lstset{showstringspaces=false}          % don't mark spaces in strings

\renewcommand{\dateseparator}{-}

% command for turning indent back on after \flushleft
\newcommand{\indenton}{\RaggedRight\parindent=15pt}

% command for vertical space
\newcommand{\vspacesmall}{\vspace{3mm}}
\newcommand{\vspacebig}{\vspace{6mm}}

% style for code inlined in text:
\newcommand{\codei}[1]{\bfseries \ttfamily{#1}\normalfont}


\begin{document}

\begin{titlepage}
    \centering
   
    \null %empty box needed for vfill to work
    \vfill

   {\bfseries\Huge
    Vector3d.h
    \vspacesmall
    
    3-dimensional vector extension for 
    \vspacesmall
        
    C++ vector class library 
    \vspacebig
        
   }        
    \vspacebig
    
   {\Large    
    Agner Fog
    \vspacebig
    
    \copyright\ \today. Apache license 2.0
   }
    
    \vfill
    
    \includegraphics[width=306pt]{freesoftwarelogo.jpg}
    \vfill
    
\end{titlepage}

\RaggedRight

\chapter{Introduction}\label{chap:Introduction}
3-dimensional vectors are useful in geometry and physics.
The file vector3d.h provides vector classes, operators, and functions for 
calculations with 3-D vectors. This is an extension to the Vector Class Library.
\vspacesmall

The classes listed below are defined. Common operators and functions are defined for these classes:

\begin {table}[H]
\caption{3-D vector classes}
\label{table:Vector3DClasses}
\begin{tabular}{|p{24mm}|p{20mm}|p{20mm}|p{22mm}|p{20mm}|p{28mm}|}
\hline
\bfseries vector class & \bfseries Precision &  \bfseries 3-D vectors per instance & \bfseries Correspon-ding real vector class & \bfseries Total bits & \bfseries Recommended minimum \newline instruction set \\ \hline
Vec3Df  & \centering single & \centering  1 & \centering Vec4f & \centering 128 & SSE2 \\ \hline
Vec3Dd  & \centering double & \centering 1 & \centering Vec4d & \centering 256 & AVX \\ \hline
\end{tabular}
\end{table}
\vspacebig



\section{Compiling} \label{Compiling}
The 3-D vector class extension to the Vector Class Library is compiled in the same way as the Vector Class Library itself. All x86 and x86-64 platforms are supported, including Windows, Linux, and Mac OS. 
The following C++ compilers can be used: Gnu, Clang, Microsoft, and Intel. 
See the Vector Class Library manual for further details.
\vspacesmall

This example shows how to use the 3-D vector classes:

\begin{example}
\label{example1}
\end{example} % frame disappears if I put this after end lstlisting
\begin{lstlisting}[frame=single]
// Example for 3-D vectors
#include <stdio.h>
#include "vectorclass.h"  // vector class library
#include "vector3d.h"     // extension for 3-D vectors

// function to print 3-D vector:
template <typename V>
void printv3 (const char * text, V a) {
    auto aa = a.to_vector();  // get elements as real vector
    printf("\n%s ", text);    // print text
    printf("(%.3G,%.3G,%.3G)", aa[0], aa[1], aa[2]);
}

int main() {
    // define 3-D vectors
    Vec3Dd a(1,2,3);                // x = 1, y = 2, z = 3
    Vec3Dd b(4,5,6);                // x = 4, y = 5, z = 6
    Vec3Dd c = a + b;               // add vectors
    Vec3Dd d = cross_product(a, b); // x-product
    double e = dot_product(a, b);   // dot-product
    // print results
    printv3("a = ", a);             // a = (1,2,3)
    printv3("b = ", b);             // b = (4,5,6)
    printv3("c = ", c);             // c = (5,7,9)
    printv3("d = ", d);             // d = (-3,6,-3)
    printf ("\ne = %f", e);         // e = 32
}
\end{lstlisting}
\vspacesmall


\chapter{Constructing 3-D vectors and loading data} 
\label{Constructing3Dvectors}

There are several ways to create 3-D vectors and put data into them. These methods are listed here.
\vspacebig

\begin{tabular}{|p{25mm}|p{100mm}|}
\hline
\bfseries Method & default constructor \\ \hline
\bfseries Defined for & all 3-D vectors classes \\ \hline
\bfseries Description & the 3-D vector is created but not initialized.\newline
The value is unpredictable \\ \hline
\bfseries Efficiency & good \\ \hline
\end{tabular}
\vspacesmall

\begin{lstlisting}[frame=none]
// Example:
Vec3Dd a;    // creates a 3-D vector
\end{lstlisting}
\vspacebig


\begin{tabular}{|p{25mm}|p{100mm}|}
\hline
\bfseries Method & Construct from x,y,z coordinates \\ \hline
\bfseries Defined for & all 3-D vectors classes \\ \hline
\bfseries Description & The parameters define the x, y, and z coordinates \\ \hline
\bfseries Efficiency & good \\ \hline
\end{tabular}
\vspacesmall

\begin{lstlisting}[frame=none]
// Example:
Vec3Dd a(1,2,3);  // a = (1,2,3) (x = 1, y = 2, z = 3)
\end{lstlisting}
\vspacebig

\begin{tabular}{|p{25mm}|p{100mm}|}
\hline
\bfseries Method & member function load(p) \\ \hline
\bfseries Defined for & all 3-D vectors classes \\ \hline
\bfseries Description & Load data from array of same precision. \\ \hline
\bfseries Efficiency & good \\ \hline
\end{tabular}
\vspacesmall

\begin{lstlisting}[frame=none]
// Example:
float a[3] = {2,5,-1};
Vec3Df b;
b.load(a);  // b = (2,5,-1)
\end{lstlisting}
\vspacebig


\begin{tabular}{|p{25mm}|p{100mm}|}
\hline
\bfseries Method & member function store(p) \\ \hline
\bfseries Defined for & all 3-D vectors classes \\ \hline
\bfseries Description & Save data into array of same precision \\ \hline
\bfseries Efficiency & good \\ \hline
\end{tabular}
\vspacesmall

\begin{lstlisting}[frame=none]
// Example:
double a[3];
Vec3Dd b(4,0,3);
b.store(a);  // a = {4,0,3}
\end{lstlisting}
\vspacebig


\begin{tabular}{|p{25mm}|p{100mm}|}
\hline
\bfseries Method & member function get\_x() \\ \hline
\bfseries Defined for & all 3-D vectors classes \\ \hline
\bfseries Description & Get the x-coordinate \\ \hline
\bfseries Efficiency & good \\ \hline
\end{tabular}
\vspacesmall

\begin{lstlisting}[frame=none]
// Example:
Vec3Dd a(1,2,3);
double b = a.get_x();  // b = 1
\end{lstlisting}
\vspacebig

\begin{tabular}{|p{25mm}|p{100mm}|}
\hline
\bfseries Method & member function get\_y() \\ \hline
\bfseries Defined for & all 3-D vectors classes \\ \hline
\bfseries Description & Get the y-coordinate \\ \hline
\bfseries Efficiency & good \\ \hline
\end{tabular}
\vspacesmall

\begin{lstlisting}[frame=none]
// Example:
Vec3Dd a(1,2,3);
double b = a.get_y();  // b = 2
\end{lstlisting}
\vspacebig

\begin{tabular}{|p{25mm}|p{100mm}|}
\hline
\bfseries Method & member function get\_z() \\ \hline
\bfseries Defined for & all 3-D vectors classes \\ \hline
\bfseries Description & Get the z-coordinate \\ \hline
\bfseries Efficiency & good \\ \hline
\end{tabular}
\vspacesmall

\begin{lstlisting}[frame=none]
// Example:
Vec3Dd a(1,2,3);
double b = a.get_z();  // b = 3
\end{lstlisting}
\vspacebig

\begin{tabular}{|p{25mm}|p{100mm}|}
\hline
\bfseries Method & member function extract(index) \\ \hline
\bfseries Defined for & all 3-D vectors classes \\ \hline
\bfseries Description & index = 0, 1, 2 give the x, y, or z-coordinate, respectively \\ \hline
\bfseries Efficiency & good \\ \hline
\end{tabular}
\vspacesmall

\begin{lstlisting}[frame=none]
// Example:
Vec3Dd a(1,2,3);
double b = a.extract(2);  // b = 3
double c = a[2];          // b = 3 (the same)
\end{lstlisting}
\vspacebig

\begin{tabular}{|p{25mm}|p{100mm}|}
\hline
\bfseries Method & member function insert(index, value) \\ \hline
\bfseries Defined for & all 3-D vectors classes \\ \hline
\bfseries Description & index = 0, 1, 2 changes the x, y, or z-coordinate, respectively \\ \hline
\bfseries Efficiency & good \\ \hline
\end{tabular}
\vspacesmall

\begin{lstlisting}[frame=none]
// Example:
Vec3Dd a(1,2,3);
a.insert(0, 8);  // a = (8, 2, 3)
\end{lstlisting}
\vspacebig


\chapter{Operators}\label{chap:Operators}

\begin{tabular}{|p{25mm}|p{100mm}|}
\hline
\bfseries Operator & + \\ \hline
\bfseries Defined for & all 3-D vectors classes  \\ \hline
\bfseries Description & Add two vectors \\ \hline
\bfseries Efficiency & good \\ \hline
\end{tabular}
\vspacesmall

\begin{lstlisting}[frame=none]
// Example:
Vec3Dd a(1,2,3);
Vec3Dd b(5,6,7);
Vec3Dd c = a + b;    // c = (6,8,10)
\end{lstlisting}
\vspacebig


\begin{tabular}{|p{25mm}|p{100mm}|}
\hline
\bfseries Operator & - \\ \hline
\bfseries Defined for & all 3-D vectors classes  \\ \hline
\bfseries Description & Subtract two vectors \\ \hline
\bfseries Efficiency & good \\ \hline
\end{tabular}
\vspacesmall

\begin{lstlisting}[frame=none]
// Example:
Vec3Dd a(11,10,9);
Vec3Dd b(5,6,7);
Vec3Dd c = a - b;    // c = (6,4,2)
Vec3Dd d = - b;      // d = (-5,-6,-7)
\end{lstlisting}
\vspacebig


\begin{tabular}{|p{25mm}|p{100mm}|}
\hline
\bfseries Operator & * \\ \hline
\bfseries Defined for & all 3-D vectors classes  \\ \hline
\bfseries Description & Multiply two vectors element by element, or one vector and one scalar of the same precision \\ \hline
\bfseries Efficiency & good \\ \hline
\end{tabular}
\vspacesmall

\begin{lstlisting}[frame=none]
// Example:
Vec3Dd a(1,2,3);
Vec3Dd b(4,5,6);
Vec3Dd c = a * b;    // c = (4,10,18)
Vec3Dd d = a * 10.0; // d = (10,20,30)
\end{lstlisting}
\vspacebig


\begin{tabular}{|p{25mm}|p{100mm}|}
\hline
\bfseries Operator & / \\ \hline
\bfseries Defined for & all 3-D vectors classes  \\ \hline
\bfseries Description & Divide a vector by a scalar of the same precision \\ \hline
\bfseries Efficiency & good \\ \hline
\end{tabular}
\vspacesmall

\begin{lstlisting}[frame=none]
// Example:
Vec3Dd a(10,20,30);
Vec3Dd b = a / 5.0; // b = (2,4,6)
\end{lstlisting}
\vspacebig


\begin{tabular}{|p{25mm}|p{100mm}|}
\hline
\bfseries Operator & == \\ \hline
\bfseries Defined for & all 3-D vectors classes  \\ \hline
\bfseries Description & Compare for equality.\newline
The result is a boolean scalar. \\ \hline
\bfseries Efficiency & good \\ \hline
\end{tabular}
\vspacesmall

\begin{lstlisting}[frame=none]
// Example:
Vec3Dd a(1, 2,3);
Vec3Dd b(1,-2,3);
bool   c = (a == b);  // c = false
\end{lstlisting}
\vspacebig


\begin{tabular}{|p{25mm}|p{100mm}|}
\hline
\bfseries Operator & != \\ \hline
\bfseries Defined for & all 3-D vectors classes  \\ \hline
\bfseries Description & Compare for not equal.\newline
The result is a boolean scalar. \\ \hline
\bfseries Efficiency & good \\ \hline
\end{tabular}
\vspacesmall

\begin{lstlisting}[frame=none]
// Example:
Vec3Dd a(1, 2,3);
Vec3Dd b(1,-2,3);
bool   c = (a != b);  // c = true
\end{lstlisting}
\vspacebig


\chapter{Mathematical functions}\label{chap:MathematicalFunctions}


\begin{tabular}{|p{25mm}|p{100mm}|}
\hline
\bfseries Function & cross\_product \\ \hline
\bfseries Defined for & all 3-D vectors classes  \\ \hline
\bfseries Description & Gives the X-product of two vectors \\ \hline
\bfseries Efficiency & medium \\ \hline
\bfseries Accuracy & Calculation of the X-product involves the calculation of sums of products. Loss of precision may occur if the result is close to zero. \\ \hline
\end{tabular}
\vspacesmall

\begin{lstlisting}[frame=none]
// Example:
Vec3Dd a(1,2,3);
Vec3Dd b(4,5,6);
Vec3Dd c = cross_product(a,b); // c = (-3,6,-3)
Vec3Dd d = cross_product(b,a); // d = (3,-6,3)
\end{lstlisting}
\vspacebig


\begin{tabular}{|p{25mm}|p{100mm}|}
\hline
\bfseries Function & dot\_product \\ \hline
\bfseries Defined for & all 3-D vectors classes  \\ \hline
\bfseries Description & Gives the dot-product of two vectors. The result is a scalar \\ \hline
\bfseries Efficiency & medium \\ \hline
\end{tabular}
\vspacesmall

\begin{lstlisting}[frame=none]
// Example:
Vec3Dd a(1,2,3);
Vec3Dd b(4,5,6);
double c = dot_product(a,b); // c = 32
\end{lstlisting}
\vspacebig


\begin{tabular}{|p{25mm}|p{100mm}|}
\hline
\bfseries Function & vector\_length \\ \hline
\bfseries Defined for & all 3-D vectors classes  \\ \hline
\bfseries Description & Gives the length of the vector (Euclidian norm) \\ \hline
\bfseries Efficiency & medium \\ \hline
\end{tabular}
\vspacesmall

\begin{lstlisting}[frame=none]
// Example:
Vec3Dd a(3,0,4);
double b = vector_length(a); // b = 5
\end{lstlisting}
\vspacebig


\begin{tabular}{|p{25mm}|p{100mm}|}
\hline
\bfseries Function & normalize\_vector \\ \hline
\bfseries Defined for & all 3-D vectors classes  \\ \hline
\bfseries Description & Divides the vector by its length to give a vector with the same direction and length one. \\ \hline
\bfseries Efficiency & medium \\ \hline
\end{tabular}
\vspacesmall

\begin{lstlisting}[frame=none]
// Example:
Vec3Dd a(3,0,4);
Vec3Dd b = normalize_vector(a); // b = (0.6, 0.0, 0.8)
\end{lstlisting}
\vspacebig


\begin{tabular}{|p{25mm}|p{100mm}|}
\hline
\bfseries Function & rotate \\ \hline
\bfseries Defined for & all 3-D vectors classes  \\ \hline
\bfseries Description & Rotates a vector by multiplying a 3x3 rotation matrix by the column vector. The first three parameters define the columns of the rotation matrix. The last parameter is the vector to rotate. \\ \hline
\bfseries Efficiency & medium \\ \hline
\bfseries Accuracy & Calculation of the rotated vector involves the calculation of sums of products. Loss of precision may occur if the result is close to zero. \\ \hline
\end{tabular}
\vspacesmall

\begin{lstlisting}[frame=none]
// Example:
Vec3Dd a(1,2,3);    // vector to rotate
Vec3Dd c0(1,0,0);   // first column of matrix
Vec3Dd c1(0,0,-1);  // second column of matrix
Vec3Dd c2(0,1,0);   // third column of matrix
Vec3Dd d = rotate(c0,c1,c2,a); // d = (1,3,-2)
\end{lstlisting}
\vspacebig


\chapter{Other functions}\label{chap:OtherFunctions}

\begin{tabular}{|p{25mm}|p{100mm}|}
\hline
\bfseries Function & to\_vector \\ \hline
\bfseries Defined for & all 3-D vectors classes \\ \hline
\bfseries Description & Convert to a vector of class Vec4f or Vec4d. \\ \hline
\bfseries Efficiency & good \\ \hline
\end{tabular}
\vspacesmall

\begin{lstlisting}[frame=none]
// Example:
Vec3Df a(1,2,3);
Vec4f  b = a.to_vector(); // b = (1,2,3,0)
\end{lstlisting}
\vspacebig


\begin{tabular}{|p{25mm}|p{100mm}|}
\hline
\bfseries Function & select \\ \hline
\bfseries Defined for & all 3-D vectors classes  \\ \hline
\bfseries Description & Choose between two vectors. \\ \hline
\bfseries Efficiency & good \\ \hline
\end{tabular}
\vspacesmall

\begin{lstlisting}[frame=none]
// Example:
Vec3Df a(1,2,3);
Vec3Df b(4,5,6);
Vec3Df c = select(true,a,b);  // c = (1,2,3)
Vec3Df d = select(false,a,b); // d = (4,5,6)
\end{lstlisting}
\vspacebig


\begin{tabular}{|p{25mm}|p{100mm}|}
\hline
\bfseries Function & to\_float \\ \hline
\bfseries Defined for & Vec3Dd  \\ \hline
\bfseries Description & Convert to lower precision. The result is a Vec3Df \\ \hline
\bfseries Efficiency & good \\ \hline
\end{tabular}
\vspacesmall

\begin{lstlisting}[frame=none]
// Example:
Vec3Dd a(1,2,3);
Vec3Df b = to_float(a);// b = (1,2,3)
\end{lstlisting}
\vspacebig


\begin{tabular}{|p{25mm}|p{100mm}|}
\hline
\bfseries Function & to\_double \\ \hline
\bfseries Defined for & Vec3Df  \\ \hline
\bfseries Description & Convert to higher precision. The result is a Vec3Dd \\ \hline
\bfseries Efficiency & good \\ \hline
\end{tabular}
\vspacesmall

\begin{lstlisting}[frame=none]
// Example:
Vec3Df a(1,2,3);
Vec3Dd b = to_double(a);// b = (1,2,3)
\end{lstlisting}
\vspacebig


\end{document}
